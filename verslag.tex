\documentclass[pdftex,12pt,a4paper]{article}
\usepackage[dutch]{babel}
\usepackage[pdftex]{graphicx}
\usepackage[margin=2.5cm]{geometry}
\usepackage{fancyhdr}
\usepackage[T1]{fontenc}
\usepackage{setspace}
\usepackage{multirow}
\usepackage{multicol}
\usepackage{mathtools}
\usepackage{amssymb}
\usepackage{caption}
\usepackage{tikz}
\usetikzlibrary{arrows,decorations.markings,shapes}
\usepackage{array}
\usepackage{booktabs}
\usepackage{multirow}
\usepackage{rotating}
\usepackage[colorlinks=true]{hyperref}
\usepackage{listings}
\usepackage{color}
\usepackage{pgffor}

\definecolor{mygreen}{rgb}{0,0.6,0}
\definecolor{mygray}{rgb}{0.5,0.5,0.5}
\definecolor{mymauve}{rgb}{0.58,0,0.82}

\lstset{ %
  backgroundcolor=\color{white},   % choose the background color; you must add \usepackage{color} or \usepackage{xcolor}
  basicstyle=\footnotesize,        % the size of the fonts that are used for the code
  breakatwhitespace=false,         % sets if automatic breaks should only happen at whitespace
  breaklines=true,                 % sets automatic line breaking
  captionpos=b,                    % sets the caption-position to bottom
  commentstyle=\color{mygreen},    % comment style
  frame=single,                    % adds a frame around the code
  keepspaces=true,                 % keeps spaces in text, useful for keeping indentation of code (possibly needs columns=flexible)
  keywordstyle=\color{blue},       % keyword style
  language=Matlab,                 % the language of the code
  numbers=left,                    % where to put the line-numbers; possible values are (none, left, right)
  numbersep=5pt,                   % how far the line-numbers are from the code
  numberstyle=\tiny\color{mygray}, % the style that is used for the line-numbers
  rulecolor=\color{black},         % if not set, the frame-color may be changed on line-breaks within not-black text (e.g. comments (green here))
  showspaces=false,                % show spaces everywhere adding particular underscores; it overrides 'showstringspaces'
  showstringspaces=false,          % underline spaces within strings only
  showtabs=false,                  % show tabs within strings adding particular underscores
  stringstyle=\color{mymauve},     % string literal style
  tabsize=2,                       % sets default tabsize to 2 spaces
}

\pagestyle{fancy}
\lhead{Capita Selecta: Artifici\"ele Intelligentie}
\chead{Project}
\rhead{2013-2014}

\def\ci{\perp\!\!\!\perp}

\tikzset{
    every node/.style={
        circle,
        draw,
        inner sep     = 0pt,
        minimum width =20 pt
    }   
}  

\begin{document}

\begin{titlepage}

\begin{center}

\begin{flushleft}

\begin{tabular}{l|l}
\multirow{4}{*}{\includegraphics[scale=0.1]{KUL.jpg}} \\ & \\
& KATHOLIEKE UNIVERSITEIT LEUVEN\\[0.5cm]
& FACULTEIT INGENIEURSWETENSCHAPPEN\\
& \\
\end{tabular}\\[8cm]
\end{flushleft}



{\LARGE \textbf{Capita Selecta: Artifici\"ele Intelligentie}}\\[1cm]
\textbf{\LARGE Heuristieken}\\[8cm]


\begin{minipage}{0.4\textwidth}
\begin{flushleft} \large
Olivier \textsc{Deckers} \\
\end{flushleft}
\begin{flushleft} \large
Matthias \textsc{van der Hallen} \\
\end{flushleft}
\end{minipage}
\begin{minipage}{0.4\textwidth}
\begin{flushright} \large
s0213127
\end{flushright}
\begin{flushright} \large
s0219692\\
\end{flushright}
\end{minipage}\\[1cm]

\end{center}

\end{titlepage}

\thispagestyle{empty}  
\enlargethispage{10\baselineskip}
\setcounter{page}{0}
\newpage
\setcounter{page}{1}

\section{Literatuurstudie}
Voor dit project werd eerst in de literatuur gezocht naar recente heuristieken voor het MDGP probleem. Hierbij werden vooral papers gevonden die gebruik maakten van de technieken \emph{tabu search}, \emph{simmulated annealing} en \emph{variable neighbourhood search}. In Palubeckis et al.\cite{Palubeckis} worden deze 3 methoden met elkaar vergeleken. Hieruit blijkt dat \emph{variable neighbourhood search} het effici\"enste was voor grotere problemen. Daarom werd besloten dieper in te gaan op deze methode. Hiervoor werd de paper van Urosevic\cite{Urosevic} gebruikt. In deze paper wordt VNS gebruikt met 3 types neighbourhood: \emph{insertion}, \emph{swap} en \emph{3-chain}.

\begin{description}
\item[Insertion] Insertion werkt door 1 element van een groep te verplaatsen naar een andere groep, die nog niet aan het maximum aantal elementen zit.
\item[Swap] Swap werkt door twee elementen uit verschillende groepen met elkaar om te wisselen.
\item[3-Chain] De 3-chain methode kiest willekeurig 3 elementen $i, j, k$ uit 3 verschillende groepen $g_{i}, g_{j}, g_{k}$. Vervolgens wordt element $i$ in groep $g_{j}$ geplaatst, element $j$ in groep $g_{k}$ en element $k$ in groep $g_{i}$.
\end{description}

Verder implementeert de paper van Urosevic\cite{Urosevic} ook functionaliteit voor random restarts, waarbij het zoeken van een oplossing verder gezet wordt vanaf een initial solution.

\section{Real World Voorbeeld}

Het maximally diverse grouping probleem kan terug gevonden worden in de peer review die gebeurt bij publicatie van een paper. Het doel van de peer review stap is het waarborgen van de kwaliteit van verschillende papers. Hierbij is het belangrijk om zoveel mogelijk verscheidenheid te behouden bij het kiezen van de reviewers, om zo reviewers uit verschillende velden van expertise en uit verschillende opleidingen en universiteiten de paper te laten nalezen.

Om MDGP te gebruiken bij het kiezen van peer reviewers is een preprocessing stap nodig die de matrix van afstanden tussen de verschillende onderzoekers opstelt. Het is mogelijk om hierbij te kijken naar de thuis-universiteit en het expertise gebied van elk van de reviewers. Hoe meer de thuis-universiteiten geografisch of methodologisch bij elkaar horen, hoe kleiner de afstand tussen de researchers in de afstandsmatrix. De definitie van het MDGP probleem staat ons toe om te bepalen hoeveel reviewers er minimum en maximum aan elke paper toegekend mogen worden.

\begin{thebibliography}{9}
\bibitem{Palubeckis}
  G. Palubeckis et al.
  \emph{Comparative performance of three metaheuristic approaches
for the maximally diverse grouping problem}.
  Information Technology And Control, 2011, Vol 40, No 4.
  
\bibitem{Urosevic}
 D. Urosevic
 \emph{Variable Neighborhood search for maximum diverse grouping problem}.
 Yugoslav Journal of Operations Research, 2014, Vol 24.

\end{thebibliography}
\end{document}