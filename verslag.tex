\documentclass[pdftex,12pt,a4paper]{article}
\usepackage[dutch]{babel}
\usepackage[pdftex]{graphicx}
\usepackage[margin=2.5cm]{geometry}
\usepackage{fancyhdr}
\usepackage[T1]{fontenc}
\usepackage{setspace}
\usepackage{multirow}
\usepackage{multicol}
\usepackage{mathtools}
\usepackage{amssymb}
\usepackage{caption}
\usepackage{tikz}
\usetikzlibrary{arrows,decorations.markings,shapes}
\usepackage{array}
\usepackage{booktabs}
\usepackage{multirow}
\usepackage{rotating}
\usepackage[colorlinks=true]{hyperref}
\usepackage{listings}
\usepackage{color}
\usepackage{pgffor}

\definecolor{mygreen}{rgb}{0,0.6,0}
\definecolor{mygray}{rgb}{0.5,0.5,0.5}
\definecolor{mymauve}{rgb}{0.58,0,0.82}

\lstset{ %
  backgroundcolor=\color{white},   % choose the background color; you must add \usepackage{color} or \usepackage{xcolor}
  basicstyle=\footnotesize,        % the size of the fonts that are used for the code
  breakatwhitespace=false,         % sets if automatic breaks should only happen at whitespace
  breaklines=true,                 % sets automatic line breaking
  captionpos=b,                    % sets the caption-position to bottom
  commentstyle=\color{mygreen},    % comment style
  frame=single,                    % adds a frame around the code
  keepspaces=true,                 % keeps spaces in text, useful for keeping indentation of code (possibly needs columns=flexible)
  keywordstyle=\color{blue},       % keyword style
  language=Matlab,                 % the language of the code
  numbers=left,                    % where to put the line-numbers; possible values are (none, left, right)
  numbersep=5pt,                   % how far the line-numbers are from the code
  numberstyle=\tiny\color{mygray}, % the style that is used for the line-numbers
  rulecolor=\color{black},         % if not set, the frame-color may be changed on line-breaks within not-black text (e.g. comments (green here))
  showspaces=false,                % show spaces everywhere adding particular underscores; it overrides 'showstringspaces'
  showstringspaces=false,          % underline spaces within strings only
  showtabs=false,                  % show tabs within strings adding particular underscores
  stringstyle=\color{mymauve},     % string literal style
  tabsize=2,                       % sets default tabsize to 2 spaces
}

\pagestyle{fancy}
\lhead{Capita Selecta: Artifici\"ele Intelligentie}
\chead{Project}
\rhead{2013-2014}

\def\ci{\perp\!\!\!\perp}

\tikzset{
    every node/.style={
        circle,
        draw,
        inner sep     = 0pt,
        minimum width =20 pt
    }   
}  

\begin{document}

\begin{titlepage}

\begin{center}

\begin{flushleft}

\begin{tabular}{l|l}
\multirow{4}{*}{\includegraphics[scale=0.1]{KUL.jpg}} \\ & \\
& KATHOLIEKE UNIVERSITEIT LEUVEN\\[0.5cm]
& FACULTEIT INGENIEURSWETENSCHAPPEN\\
& \\
\end{tabular}\\[8cm]
\end{flushleft}



{\LARGE \textbf{Capita Selecta: Artifici\"ele Intelligentie}}\\[1cm]
\textbf{\LARGE Heuristieken}\\[8cm]


\begin{minipage}{0.4\textwidth}
\begin{flushleft} \large
Olivier \textsc{Deckers} \\
\end{flushleft}
\begin{flushleft} \large
Matthias \textsc{van der Hallen} \\
\end{flushleft}
\end{minipage}
\begin{minipage}{0.4\textwidth}
\begin{flushright} \large
s0213127
\end{flushright}
\begin{flushright} \large
s0219692\\
\end{flushright}
\end{minipage}\\[1cm]

\end{center}

\end{titlepage}

\thispagestyle{empty}  
%\enlargethispage{10\baselineskip}
\setcounter{page}{0}
\newpage
\setcounter{page}{1}

\tableofcontents
\newpage

\section{Literatuurstudie}
Voor dit project werd eerst in de literatuur gezocht naar recente heuristieken voor het MDGP probleem. Hierbij werden vooral papers gevonden die gebruik maakten van de technieken \emph{tabu search}, \emph{simulated annealing} en \emph{variable neighbourhood search}.

In Palubeckis et al.\cite{Palubeckis} worden deze 3 methoden met elkaar vergeleken. Hieruit blijkt dat \emph{variable neighbourhood search} het effici\"entste was voor grotere problemen. Bovendien was deze methode voor ons beiden nog ongekend, in tegenstelling tot simulated annealing en genetic algorithms, waar we al ervaring mee hadden.

Daarom werd besloten dieper in te gaan op de \emph{variable neighbourhood search} methode. Hiervoor werd de paper van Urosevic\cite{Urosevic} gebruikt. Deze leek voornamelijk interessant door de veelbelovende resultaten en omdat de paper, in 2014 gepubliceerd, ook erg recent was. 

\section{GVNS}
De paper beschrijft een variant van variable neighbourhood search, genaamd general variable neighbourhood search (GVNS). Deze meta heuristiek bestaat uit drie componenten:

\begin{enumerate}
\item Greedy initial solution
\item Local search
\item Shaking
\item Random restarts
\end{enumerate}

\subsection{Intensification}
\subsubsection{Greedy initial solution}
Een voor de hand liggende manier om een initi\"ele oplossing te genereren bestaat erin de elementen volledig willekeurig aan groepen toe te wijzen binnen de beperkingen van de grootte van deze groepen.

In de paper wordt er echter een greedy heuristiek gebruikt om de initi\"ele oplossing te genereren.
Elementen worden een voor een toegewezen aan de groep waarvoor de gemiddelde afstand van de elementen die al aan de groep toegewezen zijn tot het element maximaal is.

\subsubsection{Local search}
Als local search methode wordt variable neighbourhood descent gebruikt.

De paper omschrijft drie verschillende neighbourhood structuren: \emph{insertion}, \emph{swap} en \emph{3-chain}.

\begin{description}
\item[Insertion] Insertion werkt door een element van een groep die meer dan zijn minimum aantal elementen bevat te verplaatsen naar een andere groep, die minder dan zijn maximum aantal elementen bevat.
\item[Swap] Swap werkt door twee elementen uit verschillende groepen met elkaar om te wisselen.
\item[3-Chain] De 3-chain methode kiest willekeurig 3 elementen $i, j, k$ uit 3 verschillende groepen $g_{i}, g_{j}, g_{k}$. Vervolgens wordt element $i$ in groep $g_{j}$ geplaatst, element $j$ in groep $g_{k}$ en element $k$ in groep $g_{i}$.
\end{description}

Uit de experimenten in de paper blijkt dat de uitvoeringstijd van variable neighbourhood descent stijgt met een factor 150 wanneer 3-chain gebruikt wordt. 
De resulterende oplossingen blijken maar 1.7 keer beter te zijn.
Om deze reden worden enkel insertion en swap gebruikt. 

De tijd die gespaard wordt door 3-chain uit te schakelen kan vermoedelijk nuttiger aangewend worden door meerdere random restarts uit te voeren.

\subsection{Diversification}
\subsubsection{Shaking}
Als diversificatie stelt de paper een shaking procedure voor.
Deze procedure bestaat erin $k$ willekeurige swaps uit te voeren op de huidige oplossing. 
De parameter $k$ is een parameter van de procedure, die door GVNS bepaald wordt (zie later).

\subsubsection{Random restarts}
Verder implementeert de paper van Urosevic\cite{Urosevic} ook random restarts. Indien een betere oplossing niet gevonden wordt met VNS, wordt er een nieuwe initi\"ele oplossing gegenereerd met de greedy heuristiek die opnieuw geoptimaliseerd wordt.

In dit opzicht is het wel belangrijk dat de greedy heuristiek een willekeurige component bevat en niet deterministisch steeds dezelfde oplossing teruggeeft.

\section{Nabeschouwing van het artikel}
De gebruikte paper geeft een goede inleiding tot het probleem en introduceert duidelijk de verschillende componenten van hun oplossing, namelijk de greedy initial solution, variable neighbourhood descent, shaking en random restarts.
Een kleine fout in een formule nagelaten zijn ook de wiskundige onderbouwing en de code van de verschillende onderdelen duidelijk.

Wanneer deze verschillende onderdelen gecombineerd worden zijn echter niet altijd alle keuzes gespecificeerd.
Zo zijn er verschillende mogelijke manieren om de neighbourhood structuren te combineren tot een variable neighbourhood descent.

Deze verschillen in het aantal keer dat elke structuur de kans gegeven wordt om verbetering op te leveren en de greedyness van de descent. Het is namelijk mogelijk om de structuur toe te passen zodra hij verbetering oplevert, maar er kan ook gekozen worden om de structuur meerdere keren te testen en vervolgens de beste verbetering in gebruik te nemen.

Deze verschillende opties kunnen een grote invloed hebben op de kwaliteit van de gevonden oplossing.

De methodologie gebruikt bij het testen, hoeveel tijd en samples elke test kreeg en de parameters van variable neighbourhood search die hierbij gebruikt werden zijn wel erg duidelijk vermeld.

\section{Implementatie}
Zoals eerder vermeld laat de paper na de exacte implementatie van variable neighbourhood descent die gebruikt werd te beschrijven. \todo{Aanvullen}

\section{Real World Voorbeeld}

Het maximally diverse grouping probleem kan terug gevonden worden in het vormen van groepen researchers voor de peer review die gebeurt bij publicatie van een paper. Het doel van de peer review stap is het waarborgen van de kwaliteit van verschillende papers. Hierbij is het belangrijk om zoveel mogelijk verscheidenheid te behouden bij het kiezen van de reviewers, om zo reviewers uit verschillende velden van expertise en uit verschillende opleidingen en universiteiten de paper te laten nalezen.

Om MDGP te gebruiken bij het kiezen van peer reviewers is een preprocessing stap nodig die de matrix van afstanden tussen de verschillende onderzoekers opstelt. Het is mogelijk om hierbij te kijken naar de thuis-universiteit en het expertise gebied van elk van de reviewers. Hoe meer de thuis-universiteiten geografisch of methodologisch bij elkaar horen, hoe kleiner de afstand tussen de researchers in de afstandsmatrix. Op het vlak van expertise ligt de zaak iets moeilijk. Het is natuurlijk van belang om te zorgen dat elke reviewer tenminste een zekere mate aan basiskennis heeft over het onderwerp van de te reviewen paper. Dit kan bereikt worden door te vertrekken van papers met min of meer het zelfde vakgebied.

De definitie van het MDGP probleem staat ons ook toe om te bepalen hoeveel reviewers er minimum en maximum aan elke paper toegekend mogen worden. 

\newpage
\begin{thebibliography}{9}
\bibitem{Palubeckis}
  G. Palubeckis et al.
  \emph{Comparative performance of three metaheuristic approaches
for the maximally diverse grouping problem}.
  Information Technology And Control, 2011, Vol 40, No 4.
  
\bibitem{Urosevic}
 D. Urosevic
 \emph{Variable Neighborhood search for maximum diverse grouping problem}.
 Yugoslav Journal of Operations Research, 2014, Vol 24.

\end{thebibliography}

\section{Appendices}
\begin{figure}[H]
\begin{verbatim}
program GVNS(xopt, kmin, kmax, kstep, tmax, nrest)
    x := InitialSolution;
    x := VND(x);
    xopt := x;
    k := kmin;
    niter := 0;
    while RunningTime < tmax do
      x' := Shake(x, k);
      x'' := VND(x')
      if f(x'') > f(x) then
        x := x''; k := kmin;
      else
        k := k + kstep;
        if k > kmax then
          niter := niter+1;
          if niter = nrest then
            if f(x) > f(xopt) then xopt := x;
            x := InitialSolution;
            niter := 0;
          endif
          k := kmin;
        endif
      endif
    endwhile
\end{verbatim}
\caption{GVNS pseudocode}
\label{apx:gvns_pseudocode}
\end{figure}
\end{document}